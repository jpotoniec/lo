\documentclass{mwart}
\usepackage[utf8]{inputenc}
\usepackage{polski}
\usepackage{amsmath,amssymb}
\usepackage[margin=2cm]{geometry}
\usepackage{tikz}
\usepackage{diagbox}
\usepackage{graphviz}
\usepackage{multirow}
\usepackage{multicol}
\usepackage{url}
\usepackage{stmaryrd}
\usepackage{marvosym}


%nie radzi sobie z plznakami
%\usepackage{comment}
%\specialcomment{solution}{\begin{verse}}{\end{verse}}
%\excludecomment{solution}

%\newcommand{\solution}[1]{\begin{verse}#1\end{verse}}
\newcommand{\solution}[1]{\ignorespaces}

\newcommand{\nand}{\ensuremath\uparrow}
\newcommand{\nor}{\ensuremath\downarrow}
\newcommand{\xor}{\ensuremath\oplus}
\renewcommand{\iff}{\ensuremath\leftrightarrow}

\usepackage{tikz}
\usetikzlibrary{snakes,arrows,shapes}

\renewcommand{\iff}{\leftrightarrow}
\pagestyle{empty}
\linespread{1.3}
\newcommand{\ans}[1]{}
\begin{document}
\section{Składnia rachunku predykatów i teoria mnogości}
Stałe: $a, b, c$, zmienne: $x, y, z$, symbole funkcyjne: $f, g, h$, symbole predykatywne: $p, q, r$.

\begin{enumerate}
\item Alfabet grecki jest chętnie wykorzystywanym źródłem symboli matematycznych. Uzupełnij poniższą tabelę:
\begin{tabular}{lp{6cm}|lp{6cm}}
litera & nazwa & litera & nazwa \\
\hline
$\mu$ & mi & \ldots & gamma \\
$\sigma$ & \ldots & $\lambda$ & \ldots \\
\ldots & theta & \ldots & delta \\
$\omega$ & \ldots & \ldots & phi \\
\end{tabular}
	\item Zaznacz poprawne formuły rachunku predykatów
		\begin{multicols}{3}
			\begin{enumerate}
				\item $\forall x\, (p(x)\to \exists y\, q(y))$
				\item $\forall a\, (p(a)\to \exists y\, q(f(y)))$
				\item $\forall x\, (p(x)\to q(g(, y)))$
				\item $p(x)\to \exists q\exists y\, q(y)$
				\item $p(a)\to \exists y\, p(a)$
				\item $\forall x\, (p(x)\to q(x))$
				\item $\forall (x\, p(x)\to q(x))$
				\item $\forall x\, (p(x)\to) q(x)$				
				\item $\forall x\, (p(x)\to \exists x\, q(x))$
			\end{enumerate}
		\end{multicols}
		
	\item Uzupełnij tabelkę przez zaznaczenie, która kategoria pasuje do którego wyrażenia.
	
	\begin{tabular}{|l|l|l|l|l|}
	\hline	
	wyrażenie & atom & literał & klauzula & formuła \\
	\hline
	$pada(deszcz) \lor swieci(slonce)$ & & & & \\
	\hline	
	$\lnot pada(deszcz) \lor swieci(slonce)$ & & & & \\
	\hline	
	$pada(deszcz) \land swieci(slonce)$ & & & & \\
	\hline	
	$pada(deszcz)$ & & & & \\
	\hline	
	$\lnot pada(deszcz)$ & & & & \\
	\hline	
	$pada(x, y)$ & & & & \\	
	\hline
	\end{tabular}


	\item Połącz formuły i odpowiadające im odczytania w języku polskim. Uwaga: niektóre zdania nie mają odpowiadających im formuł.
			\begin{multicols}{2}
	\begin{enumerate}
	\item $pada(deszcz) \lor swieci(slonce)$
	\item $pada(deszcz) \xor swieci(slonce)$
	\item $pada(deszcz) \to swieci(slonce)$
	\item $pada(deszcz) \iff swieci(slonce)$
		\end{enumerate}
		\columnbreak
	\begin{enumerate}
	\item Pada deszcz wtedy, i tylko wtedy gdy świeci słonce.
\item Pada deszcz lub świeci słońce.
\item Pada deszcz wtedy, i tylko wtedy gdy jest mokro.
\item Pada deszcz albo świeci słonce.
\item Jeżeli pada deszcz, to świeci słonce.
\item Jeżeli pada deszcz, to jest mokro.
	\end{enumerate}	
		\end{multicols}
	\item Połącz formuły i odpowiadające im odczytania w języku polskim. Uwaga: niektóre zdania nie mają odpowiadających im formuł.
			\begin{multicols}{2}
	\begin{enumerate}
	\item $\forall x\,(szczeka(x) \to pies(x))$
	\item $\exists x\,(szczeka(x) \land \lnot pies(x))$
	\end{enumerate}
	\columnbreak
	\begin{enumerate}
	\item Istnieje coś, co szczeka i nie jest psem (inaczej: istnieje $x$, że $x$ szczeka i $x$ nie jest psem).
\item Istnieje coś, co jeżeli szczeka, to jest psem (inaczej: istnieje $x$, że jeżeli $x$ szczeka, to $x$ jest psem).
\item Wszystko szczeka i nie jest psem (inaczej: dla każdego $x$, $x$ szczeka i nieprawda, że $x$ jest psem)
\item Wszystko, co szczeka jest psem (inaczej: dla każdego $x$, jeżeli $x$ szczeka, to $x$ jest psem).
\item Istnieje coś, co szczeka i jest psem (inaczej: istnieje $x$, że $x$ szczeka i $x$ jest psem).
	\end{enumerate}
	\end{multicols}
\item (Pytanie dodatkowe) Co szczeka, a nie jest psem?	
		\item Domknij poniższe formuły.
		\begin{multicols}{2}
			\begin{enumerate}
				\item $p(f(x), a)$ (uniwersalnie)
				\item $p(x) \to q(a)$ (egzystencjalnie)
				\item $p(f(a))$ (egzystencjalnie)
				\item $p(x,y)\to (q(a)\to r(y))$ (uniwersalnie)
			\end{enumerate}
		\end{multicols}
\item Zapisz odczytanie następującej formuły rachunku predykatów:
\[ \forall x\, (wombat(x) \to \forall y\,(zjada(x,y)\iff roslina(y)) \] 
\item Zapisz formułę rachunku predykatów, odpowiadającą następującemu zdaniu: \emph{istnieje zwierzę, które je wyłącznie zwierzęta, które jedzą zwierzęta}
\item (Pytanie dodatkowe) Jakiego gatunku jest to zwierzę?
\item Niech $A\subseteq \Omega$:
\begin{enumerate}
\item $A\cup A=\ldots$
\item $A\cap A=\ldots$
\item $A\cup \Omega=\ldots$
\item $A\cap \Omega=\ldots$
\item Jeżeli $\Omega\subseteq A$, to co wiadomo o $A$?
\item Jeżeli $A\subseteq \emptyset$, to co wiadomo o $A$?
\end{enumerate}
\end{enumerate}


\end{document}
