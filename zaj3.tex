\documentclass{mwart}
\usepackage[utf8]{inputenc}
\usepackage{polski}
\usepackage{amsmath,amssymb}
\usepackage[margin=2cm]{geometry}
\usepackage{tikz}
\usepackage{diagbox}
\usepackage{graphviz}
\usepackage{multirow}
\usepackage{multicol}
\usepackage{url}
\usepackage{stmaryrd}
\usepackage{marvosym}


%nie radzi sobie z plznakami
%\usepackage{comment}
%\specialcomment{solution}{\begin{verse}}{\end{verse}}
%\excludecomment{solution}

%\newcommand{\solution}[1]{\begin{verse}#1\end{verse}}
\newcommand{\solution}[1]{\ignorespaces}

\newcommand{\nand}{\ensuremath\uparrow}
\newcommand{\nor}{\ensuremath\downarrow}
\newcommand{\xor}{\ensuremath\oplus}
\renewcommand{\iff}{\ensuremath\leftrightarrow}

\usepackage{tikz}
\usetikzlibrary{snakes,arrows,shapes}

\renewcommand{\iff}{\leftrightarrow}
\pagestyle{empty}
\linespread{1.3}
\newcommand{\ans}[1]{}
\begin{document}
\section*{Semantyka rachunku predykatów}
Stałe: $a, b, c$, zmienne: $x, y, z$, symbole funkcyjne: $f, g, h$, symbole predykatywne: $p, q, r$.

\begin{enumerate}
	\item Określ wartość logiczną poniższych formuł dla podanych interpretacji i wartościowań.
		\begin{enumerate}
			\item $\lnot(p(x) \land q(x, y)) \iff p(a)$\\
			$I=\left(\mathbb{N}, \{p(x)\mapsto x \text{ jest parzyste}, q(x, y)\mapsto x\geq y\}, \emptyset, \{a\mapsto 5\}\right) \qquad \sigma_I=\{x\mapsto 3, y\mapsto 17\}$
			\item $\lnot(p(x) \land q(x, y)) \iff p(a)$\\
			$I=\left(\{\clubsuit, \diamondsuit, \heartsuit, \spadesuit\}, \{p(x)\mapsto x \text{ jest czerwone}, q(x, y)\mapsto x \text{ jest ciemniejsze niż } y\}, \emptyset, \{a\mapsto \diamondsuit\}\right)$\\
			$\sigma_I=\{x\mapsto \clubsuit, y\mapsto \heartsuit\}$	
			\item $\forall x\exists y\,\left[((p(f(x))\to(q(f(x), y)\land r(x, y)))\to(\lnot q(a, b)\to \lnot p(g(x))))\to \lnot q(h(x, y), y)\right]$\\
			\begin{tabular}{ll|ll}
			symbol & interpretacja & symbol & interpretacja \\
			\hline
			dziedzina & talia 52 kart & $f(x)$ & dama tego samego koloru karcianego co $x$ \\
			\cline{1-2}
			$p(x)$ & $x$ jest figurą & $g(x)$ & ósemka tego samego koloru karcianego co $x$ \\
			$q(x, y)$ & $x$ ma mniejszą wartość niż $y$ & $h(x, y)$ & karta o wyższej z wartości spośród $x$, $y$ \\
			\cline{3-4}
			$r(x, y)$ & $x$ ma inny kolor karciany niż $y$ & $a$ & walet pik \\			
			& & $b$ & król kier
			\end{tabular}\\
			\emph{Uwaga: kolor karciany to jedna z czterech wartości $\clubsuit, \diamondsuit, \heartsuit, \spadesuit$. Wartość karty to zawarta na niej liczba bądź figura, sortowane zgodnie z następującym porządkiem: 2, 3, \ldots, 10, walet, dama, król, as.}
			%\item $((p\lor q)\lor r)\to((p\lor q)\land (p\lor r))$
			%\item $((p\lor q)\land (q\lor r)\land (r\lor p))\to (p\land q\land r)$
			%\item $(p\lor q)\to ((\lnot p\land q)\lor (p\land \lnot q))$
		\end{enumerate}
	\item Wymyśl po jednej interpretacji dla poniższych formuł. Tam, gdzie się da, zaproponuj również wartościowanie.
		\begin{multicols}{2}
			\begin{enumerate}
				\item $p(x, f(x)) \to \lnot q(x)$
				\item $\forall x\, [p(f(x), g(x)) \to \lnot q(x)]$
			\end{enumerate}
		\end{multicols}
	\item Wymyśl po jednym modelu dla poniższych formuł.
		\begin{multicols}{2}
			\begin{enumerate}
				\item $\forall x\exists y\, [p(f(x), y) \to \lnot q(x)]$
				\item $\forall x\, [p(f(x), a) \to \lnot q(x)]$
			\end{enumerate}
		\end{multicols}
	\item Wymyśl interpretację niebędącą modelem dla poniższych formuł:
		\begin{multicols}{2}
			\begin{enumerate}
				\item $\left[\exists x\, p(x)\right] \to p(a)$
				\item $\exists x\, \left[p(x) \to p(a)\right]$
				\item $\forall x\, \left[p(x) \to q(x)\right] \to \left[\exists x\, p(x)\to \forall x\,q(x)\right]$
				\item $\left[\exists x\, p(x)\to \forall x\,q(x)\right] \to \forall x\, \left[p(x) \to q(x)\right]$
				\item $\left[\exists x\, p(x)\to \exists x\, q(x)\right] \to \forall x\,\left[p(x)\to q(x)\right]$
				\item $\forall x\,\left[p(x)\to q(x)\right] \to \left[\exists x\, p(x)\to \exists x\, q(x)\right]$
			\end{enumerate}
		\end{multicols}
	\item Zaznacz formuły, które można zbadać pod kątem \emph{spełnialności}.
		\begin{multicols}{3}
			\begin{enumerate}
				\item $\forall x\, (p(x)\to \exists y\, q(y))$
				\item $\forall x\, (p(x)\to \exists y\, q(f(y)))$
				\item $\forall x\, (p(x)\to q(g(b, y)))$
				\item $p(x)\to \exists y\, q(y)$
				\item $p(a)\to \exists y\, q(y)$
				\item $\forall x\, (p(x)\to q(x))$
				\item $\forall x\, (p(x)\to \exists x\, q(x))$
			\end{enumerate}
		\end{multicols}
	\item Zaznacz w tabeli, które własności mają podane formuły\\
		\begin{tabular}{|l|l|l|l|l|}
		\hline
		formuła & prawdziwa & spełnialna & nieprawdziwa & niespełnialna \\
		\hline
		$\forall x\, [p(x) \lor \lnot p(x)]$ & & & &  \\
		\hline
		$\forall x\, [p(x) \land \lnot p(x)]$ & & & & \\
		\hline
		$\forall x\, [p(x) \xor \lnot p(x)]$ & & & & \\
		\hline
		$\forall x\exists y\, [(p(x,y) \to q(x,x))\iff(\lnot p(x,y) \lor q(x,x))]$ & & & & \\
		\hline		
		\end{tabular}
\end{enumerate}


\end{document}
